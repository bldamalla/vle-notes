\section{Fitting and azeotrope description}

As an example, we fitted data on direct vapor pressure measurements of
an ethanol--water mixture at \SI{30}{\celsius}. It was found that the
fitted constants have values $A_{21}=0.9459$ and $A_{12}=1.5009$.
For details on fitting see Supporting Information.

These parameters can then be used to create a pressure--composition plot
showing the total pressure as a function of liquid phase ethanol mole 
fraction (Figure~\ref{fig:press-plot}).

\begin{figure}[ht]
    \centering
    \includegraphics[scale=0.5]{figs/press.png}
    \caption{Vapor pressures as a function of liquid phase ethanol molar composition.
    A magnified section is also included to emphasize that maximal vapor pressure
    does not occur at ethanol saturation, but rather at $x_1\approx 0.95$.}
    \label{fig:press-plot}
\end{figure}

It can be seen from the magnified section of the figure that there appears to
be a maximum vapor pressure that is greater than the saturation vapor pressure
of ethanol. At this point, an increase in concentration of ethanol does not
increase the vapor pressure of the whole system
\begin{equation}\label{eq:press-derv}
    \dv{P}{x_1} = \dv{P_1}{x_1} + \dv{P_2}{x_1} = 0,
\end{equation}
where $P_1$ and $P_2$ are the partial vapor pressures of the ethanol and
water, respectively. It must also be noted that 
\[
    x_1 + x_2 = 1 \implies \dv{x_2}{x_1} = -1 
    \implies \dv{P_2}{x_2} = \dv{P_2}{x_1} \cdot \dv{x_2}{x_1} =
    -\dv{P_2}{x_1}.
\]
So that Equation~\ref{eq:press-derv} can be reexpressed as
\begin{equation}\label{eq:press-derv2}
    \dv{P}{x_1} = \dv{P_1}{x_1} - \dv{P_2}{x_2} = 0 \implies
    \dv{P_1}{x_1} = \dv{P_2}{x_2}.
\end{equation}

The Gibbs--Duhem equation for a binary mixture at a given temperature and
pressure is relates the chemical potentials of components in a phase as
\begin{equation}\label{eq:gibbs-duhem}
    n_1\dd{\mu_1} + n_2 \dd{\mu_2} = 0
\end{equation}
\cite[p.~439]{enggbook}.
In vapor--liquid equilibrium (VLE), the chemical potentials of the liquid
and gaseous phases of each component should be equal. Using $\mu = \mu^\circ
+ RT \ln(P/P^*)$, where $P^*$ is the saturation vapor pressure of a
component, we find that
\begin{equation}\label{eq:gibbs-duhem2}
    n_1 \dv{\mu_1}{x_1} + n_2 \dv{\mu_2}{x_1} = 0 \implies
    nRT \qty[ \dfrac{x_1}{P_1} \dv{P_1}{x_1} + \dfrac{x_2}{P_2} \dv{P_2}{x_1} ] = 0
    \implies \dfrac{x_1}{P_1} \dv{P_1}{x_1} = \dfrac{x_2}{P_2} \dv{P_2}{x_2}.
\end{equation}

Combining Equations~\ref{eq:press-derv2} and \ref{eq:gibbs-duhem2} we obtain
\[
    \dfrac{x_1}{P_1} = \dfrac{x_2}{P_2}
\]
From here, we can define a partial pressure fraction $y_i$ such that $y_i
\equiv P_i/P$. This transforms the equation above to
\begin{equation}\label{eq:azeotrope}
    \dfrac{x_1}{y_1} = \dfrac{x_2}{y_2} = \dfrac{x_1 + x_2}{y_1 + y_2} = 1
    \implies x_1 = y_1,\quad x_2 = y_2.
\end{equation}
This point in the phase diagram represents a point wherein the liquid
composition is the same as the vapor composition. This is the condition
for an azeotrope \cite[p.~394]{enggbook}. 
In this case, a maximum pressure azeotrope was described.
