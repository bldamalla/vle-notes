\section{Derivation of activity coefficients}

Using the excess free energy function given by the Margules activity model, 
it is possible to calculate the activity coefficients if we consider the 
following
\begin{align*}
    \frac1{RT} G^E = x_1\ln\gamma_1 + x_2\ln\gamma_2 &\implies
    \frac1{RT}n G^E = n_1\ln\gamma_1 + n_2\ln\gamma_2 \\ &\implies
    \ln\gamma_1 = \frac1{RT} \pdv{G^E}{n_1},
\end{align*}
and similarly
\[
    \ln\gamma_2 = \frac1{RT} \pdv{G^E}{n_2}.
\]

So that an interpretation of the activity coefficients in non-ideal mixtures is
that they are contributions to excess free energies as amounts of components change.
With this, it is then reasonable to assume that activity coefficients depend on
the mole fraction of components in a mixture.

Plugging in the excess function (Equation~\ref{eq:margules-two-param}) into the
equations above, we can obtain $\ln\gamma_1$
\stepcounter{equation}
\begin{align*}\label{eq:margules-act-1}
    \ln\gamma_1 &= \pdv{n_1} \qty[\dfrac{A_{21}n_1n_2^2}{n^2} + 
    \dfrac{A_{12}n_1n_2^2}{n^2}] \\
    &= \dfrac{2A_{21}n_1n_2n^2}{n^4} - \dfrac{2A_{21}n_1^2n_2}{n^4} +
    \dfrac{A_{12}n_2^2n^2}{n^4} - \dfrac{2A_{12}n_1n_2^2n}{n^4} \\
    &= 2A_{21}x_1x_2(1-x_1) + A_{12}x_2^2 - 2A_{12}x_1x_2^2 \\
    &= x_2^2 \qty[A_{12} + 2(A_{21} - A_{12})x_1]. \tag{\theequation}
\end{align*}
Similarly,
\begin{equation}\label{eq:margules-act-2}
    \ln\gamma_2 = x_1^2 \qty[A_{21} + 2(A_{12} - A_{21})x_2].
\end{equation}
As mentioned, the activity coefficients of the components depend on the mole 
fractions of the component in the liquid phase.

By subtracting the two equations we obtain the logarithm of the ratio of activity
coefficients of the two components of the system at a given composition, temperature
and pressure
\begin{equation}\label{eq:act-diff}
    \ln\dfrac{\gamma_1}{\gamma_2} = A_{12}x_2^2 - A_{21}x_1^2 - 2x_1x_2
    (A_{21} - A_{12})(x_2 - x_1).
\end{equation}

In practice, it is sometimes easier to calculate the parameters $A_{xx}$ based on
measurements of the activity coefficients from vapor--liquid equilibrium (VLE) data 
using Equation~\ref{eq:act-diff} and this was done in collections of \citeA{haladat}.

However, in the cited collection above, the expression for 
$\ln\frac{\gamma_1}{\gamma_2}$ used to fit the parameters is slightly different
\[
    \ln\dfrac{\gamma_1}{\gamma_2} = \ln 10 \cdot
    \log \dfrac{\gamma_1}{\gamma_2} = A_{12}x_2^2 - A_{21}x_1^2 - 2x_1x_2
    (A_{12} - A_{21}).
\]

The book by \citeA[pp.~430]{enggbook} also suggests a method to calculate $A_{xx}$
based on the activity coefficients, which are based on directly solving
Equations~\ref{eq:margules-act-1} and \ref{eq:margules-act-2} yielding:
\[
    A_{12} = \qty(2 - \frac1{x_2}) \dfrac{\ln\gamma_1}{x_2} + 
    \dfrac{2\ln\gamma_2}{x_1},\qquad
    A_{21} = \qty(2 - \frac1{x_1}) \dfrac{\ln\gamma_2}{x_1} + 
    \dfrac{2\ln\gamma_1}{x_2}.
\]
These expressions can be used to calculate the coefficients using \textit{any} point
on phase diagrams, as long as activity coefficients can be derived. It must be noted,
however that values calculated at a single point may not necessarily 
reflect values for the whole range due to experimental errors.

With this, we employed nonlinear curve fitting methods to calculate $A_{xx}$ from 
constant-temperature VLE data from literature (primarily those in a collection of 
\citeA{wichtdat}). These were then used to describe systems at those temperatures.
