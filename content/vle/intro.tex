\section{Introduction}

%% TODO: Introduction to the section

%% COMM: put here some comments about how azeotropy is applied to
%% COMM: physical systems
%% COMM: i think the thing on the energy part of separation is an
%% COMM: interesting account

Separation operations in chemical engineering processes tend
to account to most of the total energy costs of a plant \cite{seppaper}. As
it turns out, azeotropes (discussed later) in sufficiently non-ideal
mixtures can make separation of mixtures difficult.
Thermodynamic treatment of non-ideal solutions is slightly different since
there are complications to state functions brought about by interactions of
solution components, which are not treated in ideal solutions.

One way the Gibbs free energy of mixing is accounted for is by introducing
excess molar free energy functions. One such model is the Margules activity model
introduced in the late 19th century. A simplified expression of the model
is the following two-parameter form for a binary system \cite[pp.~430]{enggbook}
\begin{equation}\label{eq:margules-two-param}
    \dfrac1{RT} G^E = x_1x_2 \qty[A_{21}x_1 + A_{12}x_2],
\end{equation}
where the constants $A_{21}$ and $A_{12}$ are usually derived from vapor--liquid
equilibrium experiments. These experiments show compositions of liquids in a
solution and corresponding vapor compositions at given temperatures or pressures.

In this account, we present a demonstration of the utility of the Margules
activity model in explaining some important physical phenomena that are
consequences of deviations from the ideal behavior of fluids primarily
focusing on azeotropy.

As an example, we worked on the ethanol--water binary mixture as it provides
an interesting demonstration of the concepts.

%% TODO: add some description of azeotropes in ethanol--water systems.
