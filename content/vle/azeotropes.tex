\section{More on azeotropes}

\paragraph{Assumptions}
So far, the introduction of the concept of azeotropes using the Margules
activity model rests upon two assumptions. A first assumption is that it
is physical in the sense that it agrees with the Gibbs--Duhem equation
(Equation~\ref{eq:gibbs-duhem}). As it turns out, commonly used activity 
models follow the relation \cite[p.~440]{enggbook}. Also, for an
activity model to make sense, it must indeed follow the relation.

The second assumption so far is that the vapor pressure is a continuous function
of composition. This means that there is no liquid--liquid phase separation
in all compositions---liquids are miscible.

\paragraph{Calculation from constants}
With Equation~\ref{eq:azeotrope} at the azeotrope, we have that
\[
    y_1P = x_1\gamma_1P_1^* \implies \ln\gamma_1 = \ln \dfrac{P}{P_1^*}.
\]
Similarly
\[
    \ln\gamma_2 = \ln \dfrac{P}{P_2^*}.
\]
Subtracting these two equations we have
\begin{equation}\label{eq:act-diff2}
    \ln \dfrac{\gamma_1}{\gamma_2} = \ln \dfrac{P_2^*}{P_1^*}.
\end{equation}

Direct comparison with Equation~\ref{eq:act-diff} gives
\begin{equation}\label{eq:azeo-master}
    \ln \dfrac{P_2^*}{P_1^*} = A_{12}x_2^2 - A_{21}x_1^2 - 2x_1x_2
    (A_{21} - A_{12})(x_2 - x_1),
\end{equation}
which is a polynomial (cubic) equation in $x_1$. In principle, this is
solvable if there is sufficient information: $A_{xx}$ and
saturation vapor pressures of the components are known. For the case of
the ethanol--water mixture at \SI{30}{\celsius}, it occurs at $x_1 = 0.9476$.

Another form of Equation~\ref{eq:azeo-master} can be obtained if we let
\[
    \alpha = 2A_{21} - A_{12},\qquad \beta = -2(A_{21} - A_{12}),\qquad
    x_2 = 1 - x_1
\]
in Equations~\ref{eq:margules-act-1} and \ref{eq:margules-act-2} as
\begin{equation}\label{eq:azeo-master2}
    \ln \dfrac{P_2^*}{P_1^*} = \alpha[1 - 2x_1] + \beta[\frac32 x_1^2 - 
    3x_1 + 1],
\end{equation}
which is a quadratic equation in $x_1$.
