\section{More on azeotropes}

\paragraph{Assumptions}
So far, the introduction of the concept of azeotropes using the Margules
activity model rests upon two assumptions. A first assumption is that the
activity model is physical such that it agrees with the Gibbs--Duhem equation
(Equation~\ref{eq:gibbs-duhem}). As it turns out, commonly used activity 
models follow the relation \cite[p.~440]{enggbook}. Also, for an
activity model to make sense, it must indeed follow the relation.

The second assumption so far is that the vapor pressure is a continuous function
of composition. This means that there is no liquid--liquid phase separation
in all compositions---liquids are miscible.

\paragraph{Calculation from constants}
With Equation~\ref{eq:azeotrope} at the azeotrope, we have that
\[
    y_1P = x_1\gamma_1P_1^* \implies \ln\gamma_1 = \ln \dfrac{P}{P_1^*}.
\]
Similarly
\[
    \ln\gamma_2 = \ln \dfrac{P}{P_2^*}.
\]
Subtracting these two equations we have
\begin{equation}\label{eq:act-diff2}
    \ln \dfrac{\gamma_1}{\gamma_2} = \ln \dfrac{P_2^*}{P_1^*}.
\end{equation}

Direct comparison with Equation~\ref{eq:act-diff} gives
\begin{equation}\label{eq:azeo-master}
    \ln \dfrac{P_2^*}{P_1^*} = A_{12}x_2^2 - A_{21}x_1^2 - 2x_1x_2
    (A_{21} - A_{12})(x_2 - x_1),
\end{equation}
which is a polynomial (cubic) equation in $x_1$. In principle, this is
solvable if there is sufficient information: $A_{xx}$ and
saturation vapor pressures of the components are known. For the case of
the ethanol--water mixture at \SI{30}{\celsius}, it occurs at $x_1 = 0.9476$
with the azeotropic pressure as \SI{78.195}{\mmHg}.

Another form of Equation~\ref{eq:azeo-master} can be obtained if we let
\[
    \alpha = 2A_{21} - A_{12},\qquad \beta = -2(A_{21} - A_{12}),\qquad
    x_2 = 1 - x_1
\]
in Equations~\ref{eq:margules-act-1} and \ref{eq:margules-act-2} followed by
subtraction. Using the substitutions we obtain,
\begin{equation}\label{eq:azeo-master2}
    \ln \dfrac{P_2^*(T)}{P_1^*(T)} = \alpha[1 - 2x_1] + \beta\qty[\frac32 x_1^2
    -3x_1 + 1],
\end{equation}
which is a quadratic equation in $x_1$ and can be solved directly.

\paragraph{Other azeotropes and observations}

%% TODO: ADD HERE THE OTHER PLOTS AND OTHER COMMENTS
Using the fitted values for $A_{xx}$ in Table~\ref{tab:fits},
pressure--composition plots were also made for higher temperatures
(Figure~\ref{fig:press-high}).

\begin{figure}[ht]
    \centering
    \includegraphics[scale=0.4]{./figs/pressT50.png}
    \includegraphics[scale=0.4]{./figs/pressT90.png}
    \includegraphics[scale=0.4]{./figs/pressT150.png}
    \caption{Vapor pressures at constant temperatures as functions of 
    liquid ethanol composition. Upper left: \SI{50}{\celsius}; upper
    right: \SI{90}{\celsius}; bottom: \SI{150}{\celsius}. It can be seen
    that there are azeotropes based on the model.
    Calculated azeotropes are around $x_1 \approx 0.93, 0.95, 0.84$, respectively.}
    \label{fig:press-high}
\end{figure}

Upon closer inspection of the plot for $T=\SI{90}{\celsius}$ in 
Figure~\ref{fig:press-high}. It can be seen that the composition at 
maximal vapor pressure from data ($x_1\approx 0.865$) does not agree well 
with the calculated composition by the activity model. This is can be 
attributed to errors from fitting. Also, by inspection the azeotrope ethanol 
composition ought to be lower than calculated.

By solving Equation~\ref{eq:azeo-master2}, azeotrope compositions $x_{az}$
can be obtained (Table~\ref{tab:azeo}) and used to calculate azeotropic data
such as pressure $P_{az}$ and temperature $T_{az}$.

\begin{table}[ht]
    \centering
    \caption{Calculated azeotropic data for ethanol--water binary mixture
    using the activty model. $T_{az}$ and $P_{az}$ are the azeotropic
    temperatures and pressures, respectively. The compositions $x_{az}=
    y_{az}$ are of ethanol. The last entries are of an observed azeotrope
    for the ethanol--water mixture at atmospheric pressure.}
    \medskip
    \begin{tabular}{ccc}
        $T_{az}$ (\si{\celsius}) & $P_{az}$ (\si{\mmHg}) & $x_{az}$ \\ \hline
        \num{30} & \num{78.195} & \num{0.9476} \\
        \num{50} & \num{220.635} & \num{0.9294} \\
        \num{90} & \num{1189.652} & \num{0.9447} \\
        \num{150} & \num{7425.300} & \num{0.8412} \\
        \hline
        \num{78.17} & \num{760} & \num{0.904}
    \end{tabular}
    \label{tab:azeo}
\end{table}

A general decrease of ethanol composition at the azeotrope can be generally
observed as the temperature increases. It is also the case that azeotropic
pressure increases with temperature. From these, it is expected that since
atmospheric pressure is much closer to \SI{221}{\mmHg} than to \SI{1190}
{\mmHg}, then the azeotrope temperature is closer to \SI{50}{\celsius} than
to \SI{90}{\celsius}. The same can be said of ethanol composition.

Azeotrope data \cite[p.~331]{crc} in Table~\ref{tab:azeo} at atmospheric pressure
seems to agree with this argument.
