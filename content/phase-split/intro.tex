\section{Introduction}

%% TODO: introduce a scenario where phase splitting is used ?
%% TODO: introduce the margules activity model for binary systems and
%% TODO: truncate it

In ideal binary mixtures of the liquid pair $\{1,2\}$, the molar Gibbs free energy of 
mixing can be expressed as
\[
    \Delta G_{mix} = RT[x_1\ln x_1 + x_2\ln x_2],
\]
where $x_1$ and $x_2 = 1 - x_1$ are the respective mole fractions of liquid 
$1$ and liquid $2$, respectively.

In real systems, however, solutions vary from ideal case suggesting that the
Gibbs free energy of a system is different from that of an ideal solution.
To remedy this, mixing rules and excess Gibbs free energy functions $G^E$ 
are introduced.

One such mixing model was developed by Margules in 1895. %% TODO: add sauce
The excess Gibbs free energy $G^E$ is as follows
\begin{equation}
    \dfrac{G^E}{RT} = x_1x_2 \qty[A_{21}x_1 + A_{12}x_2] + x_1^2x_2^2
    \qty[B_{21}x_1 + B_{12}x_2] + \ldots,
\end{equation}
where constants $A, B, \ldots$ are derived from experiments. %% TODO: source

Usually, the free energy expansion is truncated to only include the first term,
resulting to the following functional form
\begin{equation}
    G^E = RTx_1x_2 \qty[A_{21} x_1 + A_{12}x_2].
\end{equation}
Through some manipulations we have that
\begin{equation}\label{eq:inf-dil}
    A_{21} = \ln \gamma_2^\infty,\qquad
    A_{12} = \ln \gamma_1^\infty,
\end{equation}
which are the activity coefficients at the infinite dilution limit for the
liquid pair $\{1,2\}$.

This article provides an exposition of the effects of asymmetric parameters
$A_{21} \neq A_{12}$ in liquid--liquid phase splitting for the Margules 
activity model in addition to a description of phase separation in the 
symmetric cases $A_{21} = A_{12}$ and $L_{21} = L_{12}$.
